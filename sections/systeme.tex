\section{Systeme von Differentialgleichungen}
	\begin{tabularx}{\columnwidth}{p{3cm}XX}
		\hline 
		\multicolumn{3}{c}{\textbf{Lineare Syteme}}\\
	lineares System von DGL 1. Ordnung &		
	$\begin{matrix} \dot{x_1} &=& a_{11}x_1 &+&\dots&+&a_{1n}(t)x_n&+&b_1(t)\\ 
	                \vdots    & & \vdots    & &     & &\vdots      & &\vdots \\
	                \dot{x_n} &=& a_{n1}x_1 &+&\dots&+&a_{nn}(t)x_n&+&b_n(t)\\ \end{matrix}$&\\

	Matrix-Vektor Notation & $\bm{\dot x} = A(t)\bm x + \bm b(t)$ & \\
	allgemeine Lösung & $\bm x = \bm x_h + \bm x_s$ & \\
	\hdashline 
	\multicolumn{3}{c}{\textbf{Analytische Verfahren}}\\
	\hdashline 
	\multicolumn{3}{c}{Homogene lineare Systeme}\\
	\multicolumn{3}{p{\columnwidth}}{$\bm{\dot x} = A \bm x$ besitzt n linear unabhängige Lösungen $\bm x_1,\dots, \bm x_n$. Diese Lösungsmenge nennt man Fundamentalsysem von Lösungen}\\
	Lösungsansatz & $\bm x_h = e^{\lambda t}\cdot \bm c \qquad \bm c = (c_1,\dots,c_n)^t \newline A\bm c = \lambda \bm c \qquad \bm c(A-\lambda I) = |A-\lambda I| = 0$ &
	$\lambda_i$: Eigenwerte von A \newline $c$: Eigenvektor von A\\
	
	Eigenwerte & \multicolumn{2}{p{14cm}}
	{(1) n linear unabhängige Reelle Eigenwerte: $\lambda_i \in \mathbb{R}\quad  \forall i$ \newline 
	 (2) mindestens ein Paar  von konjugiert komplexen Eigenwerten: $\lambda_i \in \mathbb{C}$\newline
	 (3) mehrfache Eigenwerte: $\lambda_i = \lambda_j$}\\
	(1) reeller EW & $\bm x(t) = e^{\lambda t}\bm c$ & \\
	(2) komplexer EW & $\lambda = \mu + j\nu \bm c = \bm a + j \bm b \newline \bm x(t) = e^{\lambda t}\bm c = e^{(\mu + j\nu )t}(\bm a + j\bm b)$ &
	 $\bm z_1(t) = e^{\mu t}(\bm a \cos(\nu t) - \bm b sin(\nu t))$\newline 
	  $\bm z_2(t) = e^{\mu t}(\bm a \cos(\nu t) + \bm b sin(\nu t))$\\
	(3) k-fach reelle EWs & 
	$\bm x_i(t) = e^{\lambda t} \bm p_{m-1}(t) \quad (2\leq m \leq k)$ & 
	$\bm p_m(t) = \begin{pmatrix} p_1^{m}(t) \\ \vdots \\ p_n^{m} (t) \\\end{pmatrix} \quad (0\leq m\leq k-1)$\\
	verallgemeinerter EV& $\bm p_i(t) = (t\bm v + \bm p^*) \qquad (A-\lambda I)\bm p^* =  \bm v$ & $\bm v$: schon gefundener EV $\bm p_{i-1}$\\
	(3*) k-fach komplexe EWs & Kombination aus (2) und (3) &\\
	
	
	Entkopplung & $A = TDT^{-1} \newline  D = \operatorname{diag} (\lambda_1, \dots, \lambda_n), \quad T = (\bm v_1,\dots,\bm v_n)$ \newline $\bm \dot x = A\bm x \implies T^{-1}\bm{\dot x} = DT^{-1}\bm x$& $A$: muss Diagonalisierbar sein\newline $\bm v_i$ Eigenwektoren\\
	Matrix-Exponential & $\bm x = e^{At}\bm x_0$ & $e^{At} = I + At + \dfrac{A^2t^2}{2!} + \dots = \sum\limits_{k=1}^{\infty}A^k\dfrac{t^k}{k!}$\\
	\hdashline
	Beziehung DGL höherer Ordnung &
	$x^{(n)} = f(t,x,\dot x, \ddot a, \dots, x^{n-1})\qquad \equiv$ & $\begin{matrix}
	\dot x_1 &=& x_2\\	\dot x_2 &=& x_3\\ \vdots & & \vdots \\ 	\dot x_n &=& f(t,x_1,x_2,x_3,\cdots,x_n)\\
	\end{matrix}$\\
	\hdashline 
	
	DGL n-ter Ordnung mit konst. koeffizienten & $ x^{(n)} + a_{n-1}x^{(n-1)} + \dots + a_1\dot x + a_0 x = 0$  \newline $\bm{\dot x} = A\bm x$ & 
	$A = \begin{pmatrix}
	0 & 1 & 0 & \dots & 0 \\ 0 & 0 & 1 & \dots & 0 \\ \vdots & \ddots &   & \ddots & \vdots \\ \dots &   & \dots & 0 & 1 \\ -a_0 & -a_1 &  \dots & -a_{n-2} & -a_{n-1}\\
	\end{pmatrix}$\\	
	\hdashline 
	\multicolumn{3}{c}{Inhomogene lineare Systeme}\\
	$\bm{\dot x} = A \bm x + \bm b(t)$ & Zuerst homogenes System lösen & \\
	Elimination einer Variablen	& \multicolumn{2}{l}{System DGL 1. Ordnung $\Rightarrow$  eine DGL höherer Ordnung} \\
	Ansatz vom Typ der Störunktion & Ansatz nicht komponentenweise, sondern global & \\
	Matrix-Exponentiale &
	$\bm x(t) = \int\limits_0^t e^{A(t-s)}\bm b(s) ds + e^{At}\bm c$  &
	AWP: $\bm x(t) = \int\limits_0^t e^{A(t-s)}\bm b(s) ds + e^{At}\bm x_0$\\
	\hline 
	\multicolumn{3}{c}{\textbf{Nummerische Verfahren}}\\
	\hdashline
	
	Explizites Euler-Verfahren &
	$\begin{cases} t_{k+1} &= t_k+h\\ \bm x_{k+1} &= \bm x_k + h\cdot \bm f(t_k,\bm x_k) \end{cases} $ &
	n = 2 $\begin{cases} t_{k+1} &= t_k+h\\ x_{k+1} &= x_k + h\cdot f(t_k,x_k,y_k)\\ y_{k+1} &= y_k + h\cdot g(t_k,x_k,y_k)	\end{cases}$ \\
	
	Runke-Kutta 4. Ordnung &
	$ \begin{cases} t_{k+1} &= t_k +h \\ 
	\bm m_1 &= \bm f(t_,\bm x_k)\\ 
	\bm m_2 &= \bm f(t_k+\frac{h}{2},\bm x_k+\frac{h}{2}\bm m_1)\\
	\bm m_3 &= \bm f(t_k+\frac{h}{2},\bm x_k+\frac{h}{2}\bm m_2)\\
	\bm m_4 &= \bm f(t_k+ h,\bm x_k+h\bm m_3)\\
	\bm x_{k+1} &= \bm x_k + \frac{h}{6}(\bm m_1 + 2\bm m_2 + 2\bm m_3 + \bm m_4) \end{cases}$ & \\
	\hdashline 
	DGL höherer Ordnung & \multicolumn{2}{l}{DGL höherer Ordnung muss zuerst als System von DGL 1. Ordnung formuliert werden }\\
	\hdashline 
	
	Steife Systeme & \multicolumn{2}{p{14cm}}{numerische Verfahren haben Problemen mit Systemen, welche Lösungen komplett verschiedene Grössenordnungen besitzen. Implizite Verfahren erzielen in diesem Fall bessere Resultate.}\\
	steif & $\left|\dfrac{\lambda_1}{\lambda_2}\right| \gg 1$ & $\lambda_i$: verschiedene Eigenwerte des Systems \\
	\hline 
	
	\end{tabularx}